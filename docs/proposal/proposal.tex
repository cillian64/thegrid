% ----------------------------------------
% LaTeX Article Template for CUED Reports by Jon Sowman, 2009 (jon@hexoc.com)
% ----------------------------------------

\documentclass[12pt]{article} % Set up the document class for an article
\usepackage[margin=2.5cm]{geometry}
%\usepackage{savetrees}		% Concise layout
\usepackage{amssymb}		% This packages permits using $ \therefore $
\usepackage{graphicx}		% SVG graphics includes, amongst other things
\usepackage{epstopdf}		% Allows EPS graphics includes
\usepackage{amsmath}		% This package allows the use of $ \text{} $
\usepackage{placeins}		% This package allows the use of \FloatBarrier
\usepackage{caption}		% This package allows subfigures:
\usepackage{subcaption}	% And proper subfigure captioning
\usepackage{appendix}
\usepackage[amssymb]{SIunits}

\title{\textsc{The Grid}\\\large{\textsc{Interactive Field Emission Display}}}
%\title {\textsc{emission}\\\large{\textsc{The Grid}}}
\author{David Turner (dwt27@cl.cam.ac.uk)\\Adam Greig (ag611@eng.cam.ac.uk)}
\date{} %Suppress date

% Begin the document
\begin{document}
    \maketitle

\renewcommand{\abstractname}{Summary}
\begin{abstract}
An art/engineering installation consisting a grid of poles illuminated by white LED strips.  Interactivity is provided through a computer vision system utilising a night vision camera.  Applications include a virtual maze and interactive display patterns.  Primary requirements are a $14\metre \times 18\metre$ area of unlit ground and access to mains power (approx $1.5\kilo\watt$ peak).
\end{abstract}

\begin{figure}[h]
    \centering
    \includegraphics[width=\textwidth]{pics/render1.png}
    \caption{A computer graphics simulation of \textsc{The Grid}.}
\end{figure}

\clearpage
\section{Design}
\subsection{Layout}
\textsc{The Grid} would occupy a space of approximately $18\metre \times 14\metre$.  Of this, $12\metre \times 12\metre$ is the grid itself, consisting of a $7 \times 7$ grid of poles with $2\metre$ spacing.  A backstage area holds the power and control tent as well as the camera mast.

\begin{figure}[h]
    \centering
    \includegraphics[width=7cm]{diags/plan.png}
    \caption{Plan schematic}
    \label{fig:planschematic}
\end{figure}

\subsection{Structural}
Each pole will protrude $2.5\metre$ from the ground.  The total length is $3\metre$, with $50\centi\metre$ being inserted into the ground.  The poles are constructed from $\frac{3}{4}" \times \frac{3}{4}" \times \frac{1}{16}"$ aluminium angle section.  See Appendix \ref{app:engdrawings} for detailed drawings.

The interactivity camera will be mounted $8\metre$ above the ground on a $10\metre$ fishing pole, guyed for rigidity.

\subsection{Electrical and Electronics}
\subsubsection{Cabling}
Each LED strip will consume around $2\ampere$ when active.  The wiring for one strip will consist of twin core cable carrying power and return between the strip and the control tent.  Additionally, a coaxial connection will run from the interactivity camera to the control tent.

All cabling inside \textsc{The Grid} will be buried slightly below ground to avoid a trip hazard.

\subsubsection{Switching}
Each LED strip will be controlled using a BD679 Darlington pair as a driver.  The drivers will be switched by six 8-output shift registers, themselves controlled by the CPU.

\subsubsection{Power}
The maximum power consumption of \textsc{The Grid} will be $100\ampere$.  This will be provided by four $550\watt$ ATX power supplies, each rated for $32\ampere$ on the $+12\volt$ rail.

\subsection{Software and Control}
A laptop in the control tent will generate display patterns and handle interactivity.  It will transmit lighting data via a serial link to an Arduino.  Upon receiving each frame, the Arduino will clock the data into the shift registers, then activate the output latch.

\clearpage
\begin{appendices}
\section{Engineering Drawings}
\label{app:engdrawings}

\begin{figure}[h]
    \centering
    \includegraphics[width=5cm]{diags/sidepole.png}
    \caption{Side view of a single pole}
\end{figure}

\begin{figure}[h]
    \centering
    \includegraphics[width=5cm]{diags/toppole.png}
    \caption{Top view of a single pole with LED strip}
\end{figure}

\end{appendices}

\end{document}

